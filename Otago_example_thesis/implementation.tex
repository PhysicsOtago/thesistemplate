\chapter{Implementation}
\label{chap:code}

In this chapter, a Computer Science student would be expected to
explain the design-decisions made regarding the program they have
written.  Therefore, some way of including code snippets is required,
as well as some way of printing out all of the code as an appendix.

\section{Code Snippets}
The best package for printing out code snippets is {\tt moreverb}, by
Angus Duggan.  This lets you print out code listings, with
numbered lines.  For code snippets, I suggest you use the {\tt
listing} environment, like so:

\linespread{1} \small
\begin{quote}
\begin{verbatim}
\begin{listing}[1]{1}
  ... some code here ...
\end{listing}
\end{verbatim}
\end{quote}

\linespread{1.3} \normalsize
The example above will produce a code listing with numbers every line,
starting from line 1, like this: 

\linespread{1} \small
\begin{quote}
\begin{listing}[1]{1}
#include <stdio.h>
int main(void) {
  printf("hello world!\n");
}
\end{listing}
\end{quote} 
\linespread{1.3} \normalsize

I also prefer to issue the command \verb|\linespread{1}\small| before
presenting code snippets, to condense the example somewhat.  If you do
this, start a new paragraph just before the command, otherwise it
will be applied to the paragraph above.  Also, remember to go back to
normal spacing after the code snippet by issuing the command\\
\verb|\linespread{1.3}\normalsize|.

\section{Code Dumps}
If you look at the file {\tt appendices.tex}, you will see how the
code for this document has been included as an appendix.  The
command \verb|\listinginput| has been used---it works just like
\verb|\begin{listing}| except that it reads in a file.

For the purposes of reading your code into a document, I strongly
suggest that you make a directory under your {\tt thesis} directory
called {\tt src\_links}.  Then, make symbolic links to all your code
in this directory.  Next, use those links as the arguments to
\verb|\listinginput|. Now if you change some code in a file, those
changes will be reflected in your code dump.
